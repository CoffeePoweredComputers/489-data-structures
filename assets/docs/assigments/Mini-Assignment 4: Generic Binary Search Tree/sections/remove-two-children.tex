\vspace{0.25cm}\\
\begin{minipage}{0.32\textwidth}
    \centering
    \vspace{0.1cm}
    \begin{figure}[H]
    \begin{tikzpicture}[level distance=1.5cm,
        level 1/.style={sibling distance=3cm},
        level 2/.style={sibling distance=1.5cm},
        level 3/.style={sibling distance=.75cm},
        every node/.style = {minimum width = 2em, draw, circle}
        ]
        \node {5}
            child {node {3}}
            child {node[fill=!10!orange] (R) {10}
                child {node (C) {8}
                    child {node {7}}
                    child {node {9}}
                }
                child {node (C) {12}
                    child {node {11}}
                    child {node {13}}
                }
            };
    \end{tikzpicture}
    \caption{Find the node you want to remove (orange)}
    \label{fig:twosubtrees-0}
    \end{figure}
\end{minipage}
\hfill
\begin{minipage}{0.32\textwidth}
    \centering
    \begin{figure}[H]
    \begin{tikzpicture}[level distance=1.5cm,
        level 1/.style={sibling distance=3cm},
        level 2/.style={sibling distance=1.5cm},
        level 3/.style={sibling distance=.75cm},
        every node/.style = {minimum width = 2em, draw, circle}]
        \node (P) {5}
            child {node {3}}
            child {node[fill=!10!orange] (R) {11}
                child {node {8}
                    child {node {7}}
                    child {node {9}}
                }
                child {node {12}
                    child {node[fill=red!70] (S) {11}}
                    child {node {13}}
                }
            };
        \draw[->, red, thick] (S) -- (R);
    \end{tikzpicture}
    \caption{Find the successor node (red) and copy the successor's value to the node we want to remove.}
    \label{fig:twosubtrees-1}
    \end{figure}
\end{minipage}
\hfill
\begin{minipage}{0.3\textwidth}
    \centering
    \begin{figure}[H]
    \begin{tikzpicture}[level distance=1.5cm,
        level 1/.style={sibling distance=3cm},
        level 2/.style={sibling distance=1.5cm},
        level 3/.style={sibling distance=.75cm},
        every node/.style = {minimum width = 2em, draw, circle}]
        \node (P) {5}
            child {node {3}}
            child {node[fill=!10!orange] (R) {11}
                child {node {8}
                    child {node {7}}
                    child {node {9}}
                }
                child {node {12}
                    child {edge from parent[draw=none]}
                    child {node {13}}
                }
            };
    \end{tikzpicture}
    \caption{Call the removal method on the successor node.}
    \label{fig:twosubtrees-2}
    \end{figure}
\end{minipage}
\vspace{0.25cm}\\

The process of removing a node that has two subtrees begins with finding the
node we want to remove (Figure~\ref{fig:twosubtrees-0}). We then have to find
the minimum node in the node we want to remove's right subtree; otherwise known
as the node we want to remove's ``inorder successor''. At this point the most
straightforward option to ``remove'' the orange node is to copy the data from
the inorder successor node (Figure~\ref{fig:twosubtrees-2}). Finally, we make
a recursive call to the \lstinline|remove| method to remove the successor
node.
