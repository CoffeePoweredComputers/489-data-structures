\begin{minipage}{0.54\textwidth}
\RestyleAlgo{ruled} 
\begin{algorithm}[H]
    \caption{Prim's Minimum Spanning Tree}\label{alg:prim}
    \DontPrintSemicolon
    \SetKwFunction{FPrims}{Prim}
    \SetKwProg{Fn}{Function}{}{\KwRet}
    \Fn{\FPrims{G, Source}}{
        \For{ u $\in$ G.Vertecies}{
            u.dist  $\gets \infty$\;
            u.parent $\gets$ null\;
        }
        Source.dist $\gets$ 0\;
        PQ $\gets \emptyset$ \;
        PQ.Enqueue(\textit{all vertecies})\;
        \While{PQ $\neq \emptyset$}{
            u $\gets$ PQ.RemoveMin()\;
            \For{edge $\in$ G.Adj[u]}{
                v $\gets$ eedge.dest\;
                \eIf{v $\in$ PQ AND edge.weight < v.dist}{
                    v.parent $\gets$ u\;
                    v.dist $\gets$ e.weight\;
                    PQ.Reprioritize(v)\;
                }
            }
        }
    }
\end{algorithm}
\end{minipage}
\hfill
% NOTE: Credit for this goes to -> https://texample.net/tikz/examples/prims-algorithm/
\begin{minipage}{0.45\textwidth}
\begin{figure}[H]
\centering
\begin{tikzpicture}[scale=1.8, auto,swap]
    % Draw a 7,11 network
    % First we draw the vertices
    \foreach \pos/\name in {{(0,1)/a}, {(2,1)/b}, {(4,1)/c},
                            {(0,0)/d}, {(3,0)/e}, {(2,-1)/f}, {(4,-1)/g}}
        \node[vertex] (\name) at \pos {$\name$};
    % Connect vertices with edges and draw weights
    \foreach \source/ \dest /\weight in {b/a/7, c/b/8,d/a/5,d/b/9,
                                         e/b/7, e/c/5,e/d/15,
                                         f/d/6,f/e/8,
                                         g/e/9,g/f/11}
        \path[edge] (\source) -- node[weight] {$\weight$} (\dest);
    % Start animating the vertex and edge selection. 
    \foreach \vertex / \fr in {d/1,a/2,f/3,b/4,e/5,c/6,g/7}
        \path<\fr-> node[selected vertex] at (\vertex) {$\vertex$};
    % For convenience we use a background layer to highlight edges
    % This way we don't have to worry about the highlighting covering
    % weight labels. 
    \begin{pgfonlayer}{background}
        \pause
        \foreach \source / \dest in {d/a,d/f,a/b,b/e,e/c,e/g}
            \path<+->[selected edge] (\source.center) -- (\dest.center);
        \foreach \source / \dest / \fr in {d/b/4,d/e/5,e/f/5,b/c/6,f/g/7}
            \path<\fr->[ignored edge] (\source.center) -- (\dest.center);
    \end{pgfonlayer}
\end{tikzpicture}
\caption{Prim's Minimum Spannning Tree (Source = D)}
\end{figure}
\end{minipage}
~\\
For the method \lstinline|primsMST| you should implement the algorithm
described by Algorithm~\ref{alg:prim}. Additionally, after running the 
algorithm your method should print all of the edges that make up the 
minimum spanning tree. Additionally, the following methods may be useful
when implementing this algorithm:
\begin{itemize}
    \item \lstinline|contains(E e)|: The \lstinline|Queue<E>| interface contains a queue method that can be used to determine if the queue contains an element.
    \item \lstinline|offer(E e)|: The \lstinline|Queue<E>| interface uses the \lstinline|offer| method to allow for the enqueueing of elements.
    \item \lstinline|poll(E e)|: The \lstinline|Queue<E>| interface uses the \lstinline|offer| method to allow for the dequeueing of elements.
    \item \lstinline|keySet()|: The \lstinline|Map<E>| interface has this method which will let you get all the keys from a map. For our adjacency graph, the keys are our set of vertecies. This will be useful for when you initially add all vertecies to the queue.
\end{itemize}

\textit{Hint:} After running this algorithm, the \lstinline|parents| HashMap
will contain these edges.

\subsubsection*{Example Output}

\begin{shell}
Prims Edges (Source=D):
A, D
B, A
C, E
D, null
E, B
F, D
G, E
\end{shell}
