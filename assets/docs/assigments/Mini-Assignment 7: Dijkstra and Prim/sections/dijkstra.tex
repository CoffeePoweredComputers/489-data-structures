
\begin{minipage}{0.53\textwidth}
\RestyleAlgo{ruled} 
\begin{algorithm}[H]
    \caption{Dijkstra's Shortest Path}\label{alg:dijkstra}
    \DontPrintSemicolon
    \SetKwFunction{FDijkstra}{Dijkstra}
    \SetKwProg{Fn}{Function}{}{\KwRet}
    \Fn{\FDijkstra{G, Source, Dest}}{

        \For{ u $\in$ G.Vertex}{
            u.distance  $\gets \infty$\;
            u.parent $\gets$ null\;
        }
        Source.distance $\gets$ 0\;
        PQ $\gets$ G.Vertex\;
        PQ.Enqueue(\textit{all vertecies})\;
        \While{PQ $\neq \emptyset$}{
            u $\gets$ PQ.getMin()\;
            \For{edge $\in$ G.Adj[u]}{
                v $\gets$ edge.dest\;
                PathWeight $\gets$ u.distance + edge.weight)\;
                \If{v $\in$ PQ AND PathWeight < v.distance}{
                    v.parent $\gets$ u\;
                    v.distance $\gets$ PathWeight\;
                    PQ.reprioitize(v)
                }
            }
        }
    }
\end{algorithm}
\end{minipage}
\hfill
% NOTE: Credit for this goes to -> https://texample.net/tikz/examples/prims-algorithm/
\begin{minipage}{0.5\textwidth}
\begin{figure}[H]
\centering
\begin{tikzpicture}[scale=1.5, auto,swap]
    % Draw a 7,11 network
    % First we draw the vertices
    \foreach \pos/\name in {{(0,1)/a}, {(2,1)/b}, {(4,1)/c},
                            {(0,0)/d}, {(3,0)/e}, {(2,-1)/f}, {(4,-1)/g}}
        \node[vertex] (\name) at \pos {$\name$};
    % Connect vertices with edges and draw weights
    \foreach \source/ \dest /\weight in {b/a/7, c/b/8,d/a/5,d/b/9,
                                         e/b/7, e/c/5,e/d/15,
                                         f/d/6,f/e/8,
                                         g/e/9,g/f/11}
        \path[edge] (\source) -- node[weight] {$\weight$} (\dest);
    % Start animating the vertex and edge selection. 
    \foreach \vertex / \fr in {d/1,a/2,f/3,b/4,e/5,c/6,g/7}
        \path<\fr-> node[selected vertex] at (\vertex) {$\vertex$};
    % For convenience we use a background layer to highlight edges
    % This way we don't have to worry about the highlighting covering
    % weight labels. 
    \begin{pgfonlayer}{background}
        \pause
        \foreach \source / \dest / \fr in {d/b/4,d/e/5,e/f/5,b/c/6,f/g/7}
            \path<\fr->[ignored edge] (\source.center) -- (\dest.center);
        \foreach \source / \dest in {b/d, c/b, d/a, f/e, f/d,g/f}
            \path<+->[selected edge] (\source.center) -- (\dest.center);
    \end{pgfonlayer}
\end{tikzpicture}
\caption{Dijkstra's Shortest Path (Source = D)}
\end{figure}
\end{minipage}
~\\
Though there are several variations to Dijkstra's shortest path algorithm the
one we will be focusing on will be for weighted, undirected graphs. This is in
preparation for the final project which deals with the creation and traversal
of a graph of this type. Additionally, we will be focusing on the
implementation on finding the shortest path from a given source node to a given
destination.  The first step in doing this will be to implement
Algorithm~\ref{alg:dijkstra} wich will find the shortest path from a given
source node to \textit{every} vertex in the graph. 
~\\

\RestyleAlgo{ruled} 
\begin{algorithm}[H]
    \caption{Backtrack Traversal from Dijkstra's}\label{algo:dijkstrabacktrack}
    \DontPrintSemicolon
    \SetKwFunction{FPrintDijkstra}{PrintDijkstraSP}
    \SetKwProg{Fn}{Function}{}{\KwRet}
    \Fn{\FPrintDijkstra{G, Source, Dest}}{

        Q $\gets \emptyset$\;
        curr $\gets$ Dest\;
        \While{Source $\notin$ Q}{
            Q.add(curr)\;
            curr = curr.parent\;
        }


    }
\end{algorithm}

~\\
Upon the completion of running the algorithm you should have two data
structures, one storing all of the nodes and their parents and another storing
the total distance from the source node to each vertex in the graph.  We can
use another algorithm to begin at the destination and backtrack to the source.
Along the way we can collect nodes and recreast the shortest path traversal.
The algorithm for doing so is given by Algorithm~\ref{algo:dijkstrabacktrack}.
Here, you will begin at the destination and work your way backwards through
your traversal (i.e., the parents of each vertex) until you find the source.
Once this is done, output that traversal in addition to the total distance.
The methdod you will implement for doing this is called \lstinline|printSP|
and takes: 1) the source vertex, 2) the destination vertex, 3) the map of
parent-child pairs, and 3) the map of distance-value pairs.

\subsubsection*{Example Output}

\begin{shell}
Dijkstras Edges (Source=D, Dest=G):
[D, F, G]: 17
\end{shell}
