he XML document we are given contains many many fields with various
information on each type of vehicle. We want our parse to do the following:
\begin{enumerate}
  \item Parse the data in using the XML java DOM parser.
  \item Filter out the data we are interested in and retrieve the values.
  \item Instantiate Vehicle objects with those values.
\end{enumerate}


\subsubsection{Attributes}

For our class we will need two \lstinline|private| attributes:
\begin{enumerate}
    \item \lstinline|vehicleDatabase|  \textrightarrow \  An \lstinline|ArrayList| of \lstinline|Vehicle| instances. This will contain all of the vehicles in our database.
    \item \lstinline|makes| \textrightarrow \ An \lstinline|ArrayList| of strings. This will contain a list of all the unique makes in our database.
    \item \lstinline|models| \textrightarrow \ An \lstinline|ArrayList| of string. This will contain a list of all the unique models in our database.
\end{enumerate}

\textbf{Your Task: } Create these attributes according to their
specifications. DO NOT initialize them however. We will do that later in our
constructor.

\subsubsection{Constructor and XML Parsing}

For this portion of the assignment, create a constructor that takes a single
string, \lstinline|filepath|. This file path is intended to be either an
absolute or relative path to an XML file of the same form as vehicles.xml. The
purpose of our constructor is to parse this file and use it's contents to
populate our class's attributes.

The general template for parsing an XML file in Java using \lstinline|javax| is as 
follows:
\begin{lstlisting}{\scriptsize}

DocumentBuilderFactor dbf = DocumentBuilderFactory.newInstance();

DocumentBuilder db = null;
try{
    db = dbf.newDocumentBuilder();
} catch(ParserConfigurationException e{
    //...
}

File f = new File(filepath);

Document doc = null;
try{
    doc = db.parse(f);
} catch(SAXException | IOException e){
    //...
}
\end{lstlisting}

Following this set of operations you will be left with \lstinline|dbf|. From there
you should use the \lstinline|doc.getElementsByTagName(tag)| to:
\begin{enumerate}
    \item To get a \lstinline|NodeList| of each vehicle in the file and iterate over that list. For each item in that list:
    \begin{enumerate}
        \item Get the 
    \end{enumerate}
\end{enumerate}

\subsubsection{Getters}

\textbf{Your Task:} 
The only getter you will need for this class is for the \lstinline|makes|
attribute. We will be building other query function that will allow the user to
interact with the database. Be sure to create the getter in accordance with
the specifications detailed in Section \ref{sec:getter}.

\subsubsection{Query Functions}

To interact with the ``database'' we will be implementing the following functions 

\paragraph{\lstinline|public ArrayList<String> queryClasses(String
user_make)\{ ... \}|} This function should  search the database, and return a
\textit{sorted} ArrayList of string containing all the \textit{unique} vehicle
classes that are of the specified \lstinline|user_make|. Be sure to ignore the
case of the string \lstinline|user_make| when making comparisons.

\paragraph{\lstinline|public ArrayList<String> queryModels(String user_make,
String user_class)\{ ... \}|} This function should search the database and return a
\lstinline{sorted} ArrayList of strings containing all the \textit{unique}
vehicle models present in the database that have the specified
\lstinline|user_make| and \lstinline|user_model|. Be sure to ignore the case of
the parameters when making comparisons.

\paragraph{\lstinline|public ArrayList<Vehicle> queryVehicles(String
user_model)\{ ... \}|} Search the database and return an ArrayList of all the
\lstinline|Vehicle| instances in the database that match the model specified in
the parameter \lstinline|user_model|. Be sure to ignore the case of
\lstinline|user_model| when making comparisons.
